%--------------------------
% Curriculum Vitae in Latex
% Author : Yuchong Pan
% License : MIT
%--------------------------

\documentclass[margin,line]{res}

\usepackage{pifont}
\usepackage{hyperref}
\usepackage[utf8x] {inputenc}
\usepackage{lmodern,textcomp}

%\topmargin .5in
%\oddsidemargin -.5in
%\evensidemargin -.5in
%\textwidth=6.0in
 \textheight=9.0in
%\itemsep=0in
%\parsep=0in
\usepackage{fancyhdr}
%\topmargin=0in
%\textheight=8.5in
\pagestyle{fancy}
\renewcommand{\headrulewidth}{0pt}
\fancyhf{}
%\cfoot{\thepage}
%\lfoot{\textit{\footnotesize Research Statement}}
%\rfoot{{\footnotesize Curriculum Vitae, Yuchong Pan, \thepage}}


\newenvironment{list1}{
  \begin{list}{\ding{113}}{%
      \setlength{\itemsep}{0in}
      \setlength{\parsep}{0.025in} \setlength{\parskip}{0in}
      \setlength{\topsep}{0in} \setlength{\partopsep}{0in}
      \setlength{\leftmargin}{0.17in}}}{\end{list}}
\newenvironment{list2}{
  \begin{list}{$\bullet$}{%
      \setlength{\itemsep}{0in}
      \setlength{\parsep}{0in} \setlength{\parskip}{0in}
      \setlength{\topsep}{0in} \setlength{\partopsep}{0in}
      \setlength{\leftmargin}{0.2in}}}{\end{list}}
\newenvironment{list3}{
  \begin{list}{\ding{113}}{%
      \setlength{\itemsep}{0.05in}
      \setlength{\parsep}{0.025in} \setlength{\parskip}{0in}
      \setlength{\topsep}{0in} \setlength{\partopsep}{0in}
      \setlength{\leftmargin}{0.17in}}}{\end{list}}


\begin{document}

\name{Yuchong Pan \vspace*{.1in}}

\begin{resume}

\section{\sc Contact Information}

\vspace{.05in}
\begin{tabular}{p{2in}}
+1 (604) 782-7439 \\
\href{mailto:panyuchong@gmail.com}{panyuchong@gmail.com}
\href{http://ypan.me/}{http://ypan.me}
\end{tabular}


\section{\sc Research Interests}
Algorithms, combinatorics, optimization, theoretical computer science --- especially combinatorial optimization, submodular optimization, network flow theory, network design, graph algorithms, data structures, graph theory, complexity theory.

\section{\sc Education}

{\bf University of British Columbia}\\
\vspace*{-.1in}
\begin{list1}
\item[] B.Sc., Computer Science and Mathematics, Combined Honours, expected 2021
  \begin{list2}
  \item[$\circ$] Minor in Arts, Philosophy
  %\item[$\circ$] Thesis: \emph{The Minimum-Cost Congestion of Single-Sink Unsplittable Flows} (work in progress) [\href{http://ypan.me/docs/thesis-proposal.pdf}{Proposal}]
  %\item[$\circ$] Advisor: F.\ Bruce Shepherd
  \end{list2}
\end{list1}


\section{\sc Employment}

{\bf Microsoft Corporation}\\
\vspace*{-.1in}
\begin{list1}
\item[] Software Engineer Intern, 2020
\item[] Software Engineer Intern, 2019
\item[] Software Engineer Intern, 2018
\end{list1}

{\bf University of British Columbia}\\
\vspace*{-.1in}
\begin{list1}
\item[] Undergraduate Teaching Assistant, 2020
\item[] Undergraduate Academic Assistant, 2019--2020
\item[] Undergraduate Teaching Assistant, 2019
\item[] Student Assistant, 2019
\item[] Undergraduate Teaching Assistant, 2018
\end{list1}

{\bf Jisuanke}\\
\vspace*{-.1in}
\begin{list1}
\item[] Teaching Researcher, 2018--2019
\item[] Lecturer, 2018--2019
\end{list1}

{\bf Sogou, Inc.}\\
\vspace*{-.1in}
\begin{list1}
\item[] Software Engineer Intern, 2017
\end{list1}

{\bf InitialView}\\
\vspace*{-.1in}
\begin{list1}
\item[] Software Engineer Intern, 2016--2017
\end{list1}


% \section{\sc Publications}

% J. Doe,  \textit{A simple piston problem in one dimension,}
% submitted to Nonlinearity (May 1998).


% A. Smith and J. Doe,  \textit{Semiclassical generalization
% of the Darboux-Christoffel formula,} J. Math. Phys. \textbf{43}
% (1996), no. 10, 4668-4680.


% \section{\sc Conference \\ Talks}

% \emph{A simple piston problem}, $95^ {th} $ Statistical Mechanics
% Conference, Rutgers University. (May 1996)

% \emph{A simple piston problem}, Workshop on Dynamical Systems and
% Related Topics, University of Maryland, College Park. (March 1996)


% \section{\sc Other Talks}

% \emph{The notorious piston problem and some recent results obtained
% by averaging}, S�minaire interne, �cole normale sup�rieure de Lyon,
% France. (December 1995)

% \emph{The notorious piston problem and some recent results obtained
% by averaging}, Seminar in Nonlinear Systems, Stevens Institute of
% Technology. (November 1995)

% \emph{Anosov's averaging theorem and an application}, Young Person's
% Seminar, Time at work trimester on dynamical systems, Institut Henri
% Poincar�, Paris, France. (July 1995)

% \emph{Ergodicity and averaging: A discussion of a theorem due to
% Anosov and a possible application}, Dynamical System Seminar, New
% York University. (March 1995)



% \newpage


\section{\sc Research Experience}

{\bf University of British Columbia}\\
\vspace*{-.1in}
\begin{list3}
\item[] Optimization problems on network flows with side constraints (thesis), 2020--2021
  \begin{list2}
  \item[$\circ$] Advisor: F.\ Bruce Shepherd
  \item[$\circ$] Studied several optimization problems on network flows with side constraints (i.e., unsplittable, confluent, and $d$-furcated flows), including the \emph{minimum congestion}, \emph{maximum routable demands}, and \emph{minimum number of rounds} problems.
  \item[$\circ$] Studied several classical algorithms on these optimization problems, including
  \begin{list2}
    \item the $2$-approximation algorithm for the \emph{minimum congestion} problem on single-sink unsplittable flows by Dinitz et al.\ (1999),
    \item the $O(1 + \log k)$-approximation algorithm for the \emph{minimum congestion} problem on confluent flows by Chen et al.\ (2004),
    \item the $3$-approximation algorithm for the \emph{maximum demands} problem on confluent flows by Chen et al.\ (2004), and
    \item the $(1 + \frac{1}{d - 1})$-approximation algorithm for the \emph{minimum congestion} problem on $d$-furcated flows with $d \geq 2$ by Donovan et al.\ (2007).
  \end{list2}
  \item[$\circ$] Studied several open questions relevant to these optimization problems, including Michel Goemans' $2$-congestion conjecture for the cost version of the \emph{minimum congestion} problem on single-sink unsplittable flows, $O(1)$-congestion for bifurcated flows, and $O(1)$ confluent rounds to route all demands.
  \item[$\circ$] Documents: [\href{http://ypan.me/docs/thesis-proposal.pdf}{Proposal}]
  \end{list2}
\item[] Gradual typing of recursive types, 2019--2020
  \begin{list2}
  \item[$\circ$] Advisor: Ronald Garcia
  \item[$\circ$] Applied the \emph{Abstract Gradual Typing (AGT)} approach, based upon abstract interpretation, to iso-recursive and equi-recursive types to obtain static and dynamic semantics for \emph{gradual typing} in terms of pre-existing static types, which enables programming languages to seamlessly combine dynamic and static checking. In particular, the dynamic semantics of the gradual language are induced from an internal runtime language whose terms represent corresponding source gradual typing derivations.
  \item[$\circ$] Proved that the gradual language with recursive types induced by the AGT approach satisfies the refined criteria for gradual typing of Siek et al.\ (2015), including type safety, static guarantee and dynamic guarantee.
  \item[$\circ$] Wrote a tutorial that demonstrates the AGT approach on a toy language BA (Boolean and Arithmetic Language), which produces TBA, GBA and MBA (Typed/Gradual/Mixed Boolean and Arithmetic Language, respectively).
  \end{list2}
\end{list3}


\section{\sc Teaching Experience}

{\bf University of British Columbia}\\
\vspace*{.05in}
\emph{Teaching Assistant} \\
\begin{tabular}{@{\hspace*{0.17in}}p{1in}p{4in}}
  CPSC 311 & Definition of Programming Languages, Fall 2020 \\
  CPSC 421/501 & Introduction to Theory of Computing (graduate), Fall 2019 \\
  CPSC 121 & Models of Computation, Fall 2018
\end{tabular}

\emph{Academic Assistant} \\
\begin{tabular}{@{\hspace*{0.17in}}p{1in}p{4in}}
  CPSC 411 & Introduction to Compiler Construction, Fall 2019--Spring 2020 \\
  & \emph{\small Involved in the redesign of the course, supervised by William J. Bowman.}
\end{tabular}

{\bf Jisuanke}\\
\vspace*{.05in}
\emph{Lecturer} \\
\begin{tabular}{@{\hspace*{0.17in}}p{2.25in}p{4in}}
  Competitive Programming, Level 6 & Spring 2019 \\
  \multicolumn{2}{l}{\hspace*{0.1in}\emph{\small Topics: network flows and bipartite graphs, data structures (splay tree, treap, link-cut}} \\
  \multicolumn{2}{l}{\hspace*{0.55in}\emph{\small tree), string algorithms (the Aho-Corasick algorithm, suffix array), bitmask DP,}} \\
  \multicolumn{2}{l}{\hspace*{0.55in}\emph{\small probability, computational geometry, query decomposition techniques}} \\ [0.03in]
  Competitive Programming, Level 5 & Fall 2018 \\
  \multicolumn{2}{l}{\hspace*{0.1in}\emph{\small Topics: graph connectivity, segment tree and binary indexed tree, string algorithms (the}} \\
  \multicolumn{2}{l}{\hspace*{0.55in}\emph{\small Knuth–Morris–Pratt algorithm, trie), hashing, elementary game theory, dynamic}} \\
  \multicolumn{2}{l}{\hspace*{0.55in}\emph{\small programming (tree DP, space-time optimization techniques), elementary number}} \\
  \multicolumn{2}{l}{\hspace*{0.55in}\emph{\small theory, divide-and-conquer techniques}} \\ [0.03in]
  Competitive Programming, Level 3 & Summer 2018 \\
  \multicolumn{2}{l}{\hspace*{0.1in}\emph{\small Topics: C++, dynamic programming (longest increasing subsequence, maximum subarray,}} \\
  \multicolumn{2}{l}{\hspace*{0.55in}\emph{\small longest common subsequence, edit distance, knapsack problems), search techniques}} \\
  \multicolumn{2}{l}{\hspace*{0.55in}\emph{\small (BFS, DFS, pruning, state representation)}}
\end{tabular}

\vspace{0.22in}
\emph{Teaching Researcher\vspace{.025in}} \\
\begin{tabular}{@{\hspace*{0.17in}}p{2.25in}p{4in}}
  Competitive Programming, Level 6 & Spring 2019 \\
  \multicolumn{2}{l}{\hspace*{0.1in}\emph{\small Topics: network flows and bipartite graphs, data structures (splay tree, treap, link-cut}} \\
  \multicolumn{2}{l}{\hspace*{0.55in}\emph{\small tree), string algorithms (the Aho-Corasick algorithm, suffix array), bitmask DP,}} \\
  \multicolumn{2}{l}{\hspace*{0.55in}\emph{\small probability, computational geometry, query decomposition techniques}}
\end{tabular}


\section{\sc Volunteer Experience}

{\bf THE Hack (InnoCat Technology Co., Ltd.)}\\
\vspace{-.1in}
\begin{list1}
\item[] Co-Founder, Chief Technology Officer, \& Co-Director of Corporate Relations, 2017--2018.
\end{list1}

{\bf Shaoxing No.1 High School}\\
\vspace*{-.1in}
\begin{list1}
\item[] Summer Coach (Competitive Programming), 2016
\item[] Student Lecturer (Competitive Programming), 2013--2015
\end{list1}


\section{\sc Manuscripts}

\begin{list2}
\item[$\circ$] Unsplittable Flow Problem on Paths and Trees: Closing the LP Relaxation Integrality Gap (with A.\ Jozefiak). UBC CPSC 531F Survey, 2019. [\href{http://ypan.me/docs/ufp-survey.pdf}{Link}]
\end{list2}


\section{\sc Talks and Presentations}

\begin{list2}
\item[$\circ$] Are we responsible if we have no choice? (in Chinese). No Nap Seminars. Online. 2020. [\href{http://ypan.me/docs/pap.pdf}{Slides}]
\item[$\circ$] The Single-Source Unsplittable Flow Problem. UBC Computer Science. University of British Columbia. Online. 2020. [\href{http://ypan.me/docs/ssufp-scribe.pdf}{Scribe}] [\href{http://ypan.me/docs/ssufp-note.pdf}{Note}]
%\item[$\circ$] TIL (Things I Learned): Performance Analysis of CoreCLR Interpreter. Intern Presentation, Microsoft. Online. 2020. [\href{http://ypan.me/docs/interp.pdf}{Slides}]
\item[$\circ$] Perturbation-Stable Maximum Cuts. Algorithms Reading Group, UBC Computer Science. University of British Columbia. Online. 2020. [\href{http://ypan.me/docs/maxcut.pdf}{Slides}]
\item[$\circ$] Unsplittable Flow Problem on Paths and Trees: Closing the LP Relaxation Integrality Gap (with A.\ Jozefiak). UBC CPSC 531F Survey. University of British Columbia. Vancouver, BC. 2019. [\href{http://ypan.me/docs/ufp-slides.pdf}{Slides}]
\item[$\circ$] Introduction to Communication Complexity. Quantum Club Seminar. University of California, Santa Barbara. Santa Barbara, CA. 2019.
\item[$\circ$] Gradual Typing for Octave Language (with A.\ Li, K.\ Wang, and P.\ Wang). UBC CPSC 311 Project. University of British Columbia. Vancouver, BC. 2018. [\href{http://ypan.me/docs/gradual-octave.pdf}{Report}]
\item[$\circ$] Some Math Notes (in Chinese). Competitive Programming Summer School. Shaoxing No.\ 1 High School. Shaoxing, China. 2016. [\href{http://ypan.me/docs/math.pdf}{Slides}]
\item[$\circ$] Graph Algorithms (in Chinese). Competitive Programming Summer School. Shaoxing No.\ 1 High School. Shaoxing, China. 2016. [\href{http://ypan.me/docs/graph.pdf}{Slides}]
\item[$\circ$] Miller-Rabin Primality Test and Pollard's $\rho$ Integer Factorization Algorithm (in Chinese). Competitive Programming Seminar. Shaoxing No.\ 1 High School. Shaoxing, China. 2015. [\href{http://ypan.me/docs/miller-rabin-pollard-rho.pdf}{Slides}]
\end{list2}


\section{\sc Honors and Awards}

\begin{list2}
\item[$\circ$] Faculty of Science International Student Scholarship (CAD \$7,500), University of British Columbia, 2020.
\item[$\circ$] J Fred Muir Memorial Scholarship in Science (CAD \$200), University of British Columbia, 2020.
\item[$\circ$] Trek Excellence Scholarship (CAD \$4,000), University of British Columbia, 2020.
\item[$\circ$] Science Scholar, University of British Columbia, 2020.
\item[$\circ$] Dean's Honour List, University of British Columbia, 2020.
\item[$\circ$] Faculty of Science International Student Scholarship (CAD \$5,000), University of British Columbia, 2019.
\item[$\circ$] Dean of Science Scholarship (CAD \$350), University of British Columbia, 2019.
\item[$\circ$] Trek Excellence Scholarship (CAD \$4,000), University of British Columbia, 2019.
\item[$\circ$] Stanley M Grant Scholarship in Mathematics (CAD \$1,500), University of British Columbia, 2019.
\item[$\circ$] Programming Language Implementation Summer School Fellowship (€400), 2019.
\item[$\circ$] Science Scholar, University of British Columbia, 2019.
\item[$\circ$] Dean's Honour List, University of British Columbia, 2019.
\item[$\circ$] Faculty of Science International Student Scholarship (CAD \$10,000), University of British Columbia, 2018.
\item[$\circ$] Dean of Science Scholarship (CAD \$425), University of British Columbia, 2018.
\item[$\circ$] Trek Excellence Scholarship (CAD \$4,000), University of British Columbia, 2018.
\item[$\circ$] Marie Kendall Memorial Scholarship in Science (CAD \$925), University of British Columbia, 2018.
\item[$\circ$] Joel Harold Marcoe Memorial Scholarship (CAD \$150), University of British Columbia, 2018.
\item[$\circ$] Science Scholar, University of British Columbia, 2018.
\item[$\circ$] Dean's Honour List, University of British Columbia, 2018.
\item[$\circ$] $27$th Place (out of $118$ teams), North American Invitational Programming Contest, Open Division (USA + Canada), 2018.
\item[$\circ$] $11$th Place (out of $67$ teams), ACM International Collegiate Programming Contest, Pacific Northwest Regional (Division 1), 2017.
\item[$\circ$] $1$st Place, Microsoft College Code Competition, University of British Columbia, 2017.
\item[$\circ$] $1$st Place, i-Lab Hackathon, 2017.
\item[$\circ$] $6$th Place \& Best Award for Creativity and Innovation, Unique Hackday, Huazhong University of Science and Technology, 2017.
\item[$\circ$] $2$nd Place, HackNanjing, 2017.
\item[$\circ$] Outstanding International Student Award (CAD \$6,000), University of British Columbia, 2017.
\item[$\circ$] Top $9$ \& InnoSpring Award, HACKxFDU, Fudan University, 2016.
\item[$\circ$] $1$st Place, Programming Ability Test (Advanced Division), Zhejiang University, 2016.
\item[$\circ$] Silver Medal, China Team Selection Competition for International Olympiad in Informatics, China Computer Federation, 2015.
\item[$\circ$] Bronze Medal, Asia Pacific Informatics Olympiad, China Computer Federation, 2015.
\item[$\circ$] First Prize, National Olympiad in Informatics in Provinces (Advanced Division), China Computer Federation, 2014.
\item[$\circ$] First Prize, National Olympiad in Informatics in Provinces (Advanced Division), China Computer Federation, 2013.
\end{list2}


\section{\sc Professional Service}

\emph{Journal Review} \\
\begin{tabular}{@{\hspace*{0.17in}}p{5in}}
  SIAM Journal on Discrete Mathematics (SIDMA)
\end{tabular}


\section{\sc Selected Coursework}

\emph{Mathematics\vspace{.025in}} \\
\begin{tabular}{@{\hspace*{0.17in}}p{5in}}
  Probability (graduate) \\
  Stochastic Processes (graduate) \\
  Submodular Optimization (graduate) \\
  Combinatorial Optimization (graduate) \\
  Measure Theory and Integration (graduate) \\
  Introduction to Theory of Computing (graduate) \\
  Tools for Modern Algorithm Analysis (graduate) \\
  Beyond Worst-Case Analysis (seminar) \\
  Real Variables I \& II \\
  Numerical Linear Algebra \\
  Introduction to Group Theory \\
  Introduction to Rings and Modules
\end{tabular}

\emph{Computer Science and Engineering} \\
\begin{tabular}{@{\hspace*{0.17in}}p{5in}}
  Introduction to Software Engineering \\
  Definition of Programming Languages \\
  Introduction to Compiler Construction \\
  Computer Hardware and Operating Systems \\
  Intermediate Algorithm Design and Analysis \\
  Machine Learning (Coursera, Stanford University) \\
  Programming Languages (Coursera, University of Washington CSE 341)
\end{tabular}

\emph{Philosophy} \\
\begin{tabular}{@{\hspace*{0.17in}}p{5in}}
  Metaphysics \\
  Philosophy of Law \\
  Philosophy of Religion \\
  Philosophy After 1800 (Russell \& Wittgenstein)
\end{tabular}


\section{\sc Academic Training}

\begin{list2}
\item[$\circ$] Second Programming Language Implementation Summer School. Bertinoro, Italy. 2019.
\end{list2}


\section{\sc Relevant Skills}

\vspace{.05in}
\begin{tabular}{@{}p{0.8in}p{4.25in}}

Languages:& English, Mandarin \\
Programming:& \LaTeX, Racket, Standard ML, JavaScript, C/C++, Java, C\#, Python, Ruby, MATLAB, Go, MySQL

\end{tabular}


\section{\sc Last Updated}
\today


\end{resume}

\end{document}
